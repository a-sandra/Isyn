This section is dedicated to compared Isyn simulation results to the longitudinal beam dynamics theory in very specific and well known scenarii used in accelerators. Most of the cases will be apply to the FAIR-SIS100 and also to the CERN PS machine. The following longitudinal beam simulations will be presented here, the aim being to compare the results of Isyn to the theory and observe the behaviour of the tracking code:
\begin{enumerate}
\item The simple case of a stationary single harmonic bucket, without collective effect.
\item A bunch rotation with a proton beam.
\item Simple acceleration.
\item Acceleration through transition energy.
\item A stationary single harmonic bucket with longitudinal space charge.
\item Acceleration with longitudinal space charge through transition energy.
\end{enumerate}

For each cases, an short description of the interface python main file will be given and explained as well as the theory used to compare with. Then the results of the simulations are shown. Remarks and discussions can be triggered if the  special effects or code artefacts are found.

